%
%%%%%%%%%%%%%%%%%%%%%%% file typeinst.tex %%%%%%%%%%%%%%%%%%%%%%%%%
%
% This is the LaTeX source for the instructions to authors using
% the LaTeX document class 'llncs.cls' for contributions to
% the Lecture Notes in Computer Sciences series.
% http://www.springer.com/lncs       Springer Heidelberg 2006/05/04
%
% It may be used as a temlpate for your own input - copy it
% to a new file with a new nam eand use it as the basis
% for your article.
%
% NB: the document class 'llncs' has its own and detailed documentation, see
% ftp://ftp.springer.de/data/pubftp/pub/tex/latex/llncs/latex2e/llncsdoc.pdf
%
%%%%%%%%%%%%%%%%%%%%%%%%%%%%%%%%%%%%%%%%%%%%%%%%%%%%%%%%%%%%%%%%%%%


\documentclass[runningheads,a4paper]{llncs}

\usepackage{amssymb}
\setcounter{tocdepth}{3}
\usepackage{graphicx}
\usepackage{amsmath}
\usepackage{booktabs}
%\usepackage{times}
\usepackage{perpage}
\usepackage{hyperref}
\MakePerPage{footnote}
\usepackage{multirow}
\usepackage{epstopdf} %converting to PDF
\usepackage{tikz}
\usetikzlibrary{arrows,patterns,automata,backgrounds,decorations,fit,petri,positioning,petri,shapes,calc}
\usepackage[caption=false]{subfig}
\usepackage{url}

\urldef{\mailsa}\path|{andriiro,f.m.maggi,marlon.dumas}@ut.ee, kerwin.jorbina@gmail.com|
\urldef{\mailsb}\path|dfmchiara@fbk.eu|
\urldef{\mailsc}\path|{ilya.verenich,m.larosa,simon.raboczi}@qut.edu.au|
\newcommand{\keywords}[1]{\par\addvspace\baselineskip
\noindent\keywordname\enspace\ignorespaces#1}

\begin{document}

\mainmatter  % start of an individual contribution

% first the title is needed
\title{Nirdizati: A Web-based Tool for Predictive Process Monitoring}

% a short form should be given in case it is too long for the running head
\titlerunning{Predictive Business Process Monitoring with LSTM Neural Networks}

% the name(s) of the author(s) follow(s) next
%
% NB: Chinese authors should write their first names(s) in front of
% their surnames. This ensures that the names appear correctly in
% the running heads and the author index.
%
\author{Andrii Rozumnyi\inst{1} \and Chiara Di Francescomarino\inst{2} \and Fabrizio Maria Maggi\inst{1} \and \\Ilya Verenich\inst{3,1} \and Kerwin Jorbina\inst{1} \and Marcello La Rosa\inst{3} \and \\ Marlon Dumas\inst{1} \and Simon Raboczi\inst{3}\thanks{Author names are in alphabetical order}}

\institute{University of Tartu, Estonia\\
\mailsa
\and FBK IRST, Trento, Italy\\
\mailsb
\and Queensland University of Technology, Australia\\
\mailsc
}

%
% NB: a more complex sample for affiliations and the mapping to the
% corresponding authors can be found in the file "llncs.dem"
% (search for the string "\mainmatter" where a contribution starts).
% "llncs.dem" accompanies the document class "llncs.cls".
%

\toctitle{Predictive Process Monitoring Using LSTM}
\tocauthor{Authors' Instructions}
\maketitle


\begin{abstract}
In this paper, we present a prototype of a predictive process monitoring engine for process workers and operational managers.
The developed solution, named \emph{Nirdizati}, is a configurable full-stack web application that supports users in selecting the preferred prediction method from the list of implemented methods and enables the continuous prediction of various performance indicators at runtime.
The results of the predictions, as well as the real-time summary statistics about the process execution, are presented in a dashboard that offers multiple visualization options.
The target audience of this demonstration includes process mining researchers as well as practitioners interested in exploring the potential of process monitoring.
\keywords{Process Mining, Predictive Process Monitoring, Machine Learning}
\end{abstract}


\section{Introduction} \label{sec:intro}


\section{Architectural Overview} \label{sec:arch}

\section{User Interface} \label{sec:UI}

\section{Conclusion} \label{sec:conclusion}

\bibliographystyle{splncs03}
\bibliography{paper}

\end{document}
