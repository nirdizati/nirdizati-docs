This paper discusses the integration of Nirdizati, a tool for predictive process monitoring, into the Web-based process analytics platform Apromore. Through this integration, Apromore's users can use event logs stored in the Apromore repository to train a range of predictive models, and later use the trained models to predict various performance indicators of running process cases from a live event stream. For example, one can predict the remaining time or the next events until case completion, the case outcome, or the violation of compliance rules or internal policies. %For example, in a claim handling process, the tool can predict if a claim decision will be made on time or late.
%Based on these predictions, process workers and operational managers can act proactively to resolve or mitigate potential process performance violations.
%The target audience of this demonstration includes process mining researchers as well as practitioners interested in exploring the potential of process monitoring.
%In this paper, we present a prototype of a web-based application for predictive process monitoring that can be used by process participants and operational managers to predict the future development of a currently running process execution.
%The implemented solution, named \emph{Nirdizati}, is a configurable full-stack web application that supports users in selecting the preferred prediction methods from a list of implemented algorithms and enables the continuous prediction of various measures of interest at runtime.
The predictions can be presented graphically via a dashboard that offers multiple visualization options, including a range of summary statistics about ongoing and past process cases. They can also be exported into text file for periodic reporting or to be visualized in third-parties business intelligence tools. Based on these predictions, operations managers may identify potential issues early on, and take remedial actions in a timely fashion, e.g.\ reallocating resources from one case onto another to avoid that the case runs overtime. %The target audience of this demonstration includes process mining researchers as well as practitioners interested in exploring the potential of predictive process monitoring.
